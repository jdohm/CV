\documentclass[10pt,a4paper]{scrlttr2}
\usepackage[utf8]{inputenc}
\usepackage[T1]{fontenc}
\usepackage[ngerman]{babel}
%------------------------------- Math -------------------------------------------------%
\usepackage{amsmath}
\usepackage{amsfonts}
\usepackage{amssymb}
%------------------------------- Units ------------------------------------------------%
\usepackage[per-mode=fraction,binary-units,locale=DE,complex-root-position =
before-number,scientific-notation=false,range-phrase ={\ bis\ },list-units=single,range-units=single]{siunitx}
%für das einfügen von Einheiten z.b. \si{\ohm} | per-mode=symbol setzt / anstelle von ^{-1}; fraction setzt \frac
%binary-units allows \bit \mibi etc
%locale=DE switches \num{1.5} to 1,5 ...
\sisetup{output-complex-root=\ensuremath{\mathrm{j}}}
%nutzt für imaginäre Zahlen das j, wie in der Elektrotechnik üblich
\usepackage{eurosym} % Für ein Vernünftiges Euro Zeichen mittels \euro
% € führt zu komischem Ergebnis...
\DeclareSIUnit{\EUR}{\text{\euro}} % Ermöglicht \EUR als Einheit zu nutzen
%------------------------------- Drawing ----------------------------------------------%
\usepackage{tikz}
\usetikzlibrary{3d}
\usetikzlibrary{calc}
\usepackage[european, siunitx, nooldvoltagedirection]{circuitikz}
%ermöglicht das Zeichnen von Schaltplänen in tikzpicture z.b.(0,0) to[R=$R_1$,v=$1\si{\volt}$] (2,0);
%european für europäische Symbole (z.B. Resistor)
%siunitx um mit 5<A> die korrekte Schreibweise innerhalb von Stromlaufplänen zu erhalten
\usetikzlibrary{decorations.pathmorphing}%für snake arrow
\usepackage{graphicx}% um Bilder einbinden zu können 
\usepackage{float}%für das Darstellen von figures, bildern etc. an genau diesem Ort [H]
\newcommand\arr[1][0]{node[inputarrow,rotate=#1]{}}%ermöglicht in Tikz mit dem Paket circuitikz das einfügen einer Pfeilspitze an der aktuellen Position \arr zeigt nach rechts, \arr[90] nach oben...
%------------------------------- Settings ---------------------------------------------%
%\author{Jannis Dohm}
%\title{CV - Page 1}
%------------------------------- KOMA-Settings ----------------------------------------%
\KOMAoptions{
        paper=a4,
        pagenumber=botright,
        %fromalign=right,
        %fromrule=afteraddress,
        fromphone,
%       fromfax,
        fromlogo,
        %fromurl,
        fromemail,
        backaddress,
        foldmarks,
%        headsepline,footsepline,
        enlargefirstpage
}
\setkomavar{fromname}{Jannis Dohm}
\setkomavar{fromaddress}{L\"utkendorpweg~16\\44805~Bochum}
\setkomavar{fromphone}{+49\,176\,913\,463\,54}
%\setkomavar{fromfax}{}
\setkomavar{fromemail}{dohm@posteo.net}
%\setkomavar*{emailseparator}{E-Mail}
%\setkomavar{emailseparator}{:\ }
%\setkomavar{fromurl}{}
%\setkomavar*{urlseparator}{}
%\setkomavar{urlseparator}{}
%\setkomavar{frombank}{Eine Bank\\BLZ~123\,45\,678\\Kto~123456789}
\setkomavar{place}{Bochum}
\setkomavar{fromlogo}{\parbox[b]{8cm}{\usekomafont{fromaddress}%
        {\mbox{\LARGE \bfseries Jannis Dohm}}
        \smallskip}
}
\setkomafont{backaddress}{\sffamily}
\setkomafont{fromaddress}{\sffamily}
\addtokomafont{fromname}{\scriptsize}
\addtokomafont{fromaddress}{\scriptsize}
\pagestyle{headings}
\begin{document}%\pagestyle{empty}
\begin{letter}{Test Name\\ Test Street 123\\444xx Wuppertal}
\opening{Sehr geehrter Herr}
		\closing{Mit freundlichen Grüßen}
    \begin{tikzpicture}[remember picture,overlay]
  \begin{scope}[shift={($(current page.south west)+(1.5,4.5)$)},opacity=.5]
  \draw (0,-3)node[rground]{}
  to[sqV,l={\SI{1}{\hertz}}] (0,0)
  (2.5,0) node[flipflop JK](JK1){}
  (1,2) node[vcc](vcc){VCC}
  |- (JK1.pin 1)
  (vcc) |- (JK1.pin 3)
  (0,0) - - (JK1.pin 2);
  \draw (JK1.bpin 2) node[ocirc,left]{};
  \draw (7,0) node[and port](and0){};
  \draw (7,-1) node[or port](or0){};
  \draw (7,-2) node[and port](and1){};
  \draw (7,-3) node[and port](and2){};
  \draw (or0.bout) node[ocirc,right]{};
  \draw (and1.bout) node[ocirc,right]{};
  \draw (and1.bin 1) node[ocirc,left]{};
  \draw (and2.bout) node[ocirc,right]{};
  \draw (and2.bin 2) node[ocirc,left]{};
  \begin{scope}[shift={(12,-1.45)}]
  \ctikzset{seven seg/color on=black!30, seven seg/color off=white}
  \draw (0,0)node[seven segment val=8 dot off box off]{}
  (-.75,-1.25) rectangle (.75,1.25);
  \foreach \x/\y in {1/a,2/b,3/c,4/d,5/e,6/f,7/g}{
    \draw (-.75,1.25-\x*.3) node[right]{\y}-- (-1,1.25-\x*.3) coordinate(\y);
  }
  \end{scope}
  \draw (.7,0) coordinate(cntIn) (JK1.pin 6) -- +(1.1,0)coordinate(cnt1)
  ($(cnt1|-and2.in 1)-(.5,0)$) coordinate(cnt0);
  \draw (cnt0) |- (and0.in 1)
  (cnt1) |- (and0.in 2)
  (cnt0) |- (or0.in 1)
  (cnt1) |- (or0.in 2)
  (cnt0) |- (and1.in 1)
  (cnt1) |- (and1.in 2)
  (cntIn) |- (and2.in 1)
  (cnt1) |- (and2.in 2);
  \draw (and0.out) - - +(2,0)coordinate(and0Out)
  (or0.out) - - +(2.5,0)coordinate(or0Out)
  (and1.out) - - +(2,0)coordinate(and1Out)
  (and2.out) - - +(2,0)coordinate(and2Out);
  \draw (and0Out) |- (a)
  (and0Out) |- (d)
  (or0Out) |- (b)
  (or0Out) |- (c)
  (and1Out) |- (f)
  (and2Out) |- (g)
  (e) -- +(-2.5,0) node[vcc]{} node[left]{VCC};
  %\draw (current page.north west) - - (g);
  \end{scope}
\end{tikzpicture}
\end{letter}
\end{document}